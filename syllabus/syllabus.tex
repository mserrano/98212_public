\documentclass{article}
\usepackage{fullpage}
\title{98-212: Competitive Computer Security}
\date{Fall 2013}
\author{Instructor: Maxime Serrano (mserrano@cmu.edu)}
\begin{document}
\maketitle
Welcome to 98-212, Competitive Computer Security!

This is a technical course focused on offensive computer security, particularly in the context
of the computer security competitions called ``Capture the Flag'' competitions. Its primary purpose
is to teach the tools and tricks used in offensive security, both for the purpose of honing your
skills for these competitions and also for use in better understanding what an attacker might
actually be capable of.

My name is Maxime Serrano - I am a Junior in Computer Science and Mathematical Sciences, and the 
leader of the ``Plaid Parliament of Pwning'' - which we usually refer to as the slightly-less-ridiculous
``PPP.'' PPP is Carnegie Mellon's CTF team, as well as a security research group. It is sponsored by 
professor David Brumley from the ECE department (who teaches the fantastic 18-487 course). PPP is ranked
as one of the best teams in the world, consistently taking home first place finishes, most recently
at DEFCON in Las Vegas and SECUINSIDE in Seoul, South Korea. In fact a global ranking of CTF teams 
recently ranked PPP as the best team in the world, and as the only American team in the top 10.

CTFs are competitions held both online and in-person at conferences that are competed in by many people
active in information security. Many companies send unofficial teams - for example, Raytheon SI, ManTech and
Google are represented. A few universities also have teams, such as CMU, UC Berkeley, UC Santa Barbara,
Georgia Tech, Boston University, and RPI.

This course will be held on Thursdays from 6:30 to 8 pm in Wean Hall 5403. We will cover topics ranging
throughout many aspects of computer security, including forensics, web security, cryptography,
reverse engineering and binary exploitation.

\section{Course Policies}
The StuCo system requires that I take attendance, so I will. You are allowed 2 unexcused absences, as well
as arbitrarily many ``excused absences.'' There are two primary ways to be excused. First, you can have an 
actual emergency - in which case just toss me an email and everything will be fine. Second, you can complete
the ``homework'' that is posted for that week. The homework {\em will be hard} if you do not come to lecture,
and will be {\em easy} - but still worth doing, if you're interested - if you do.

\section{Rough Schedule}
The plan is currently to follow roughly the following order:
\begin{enumerate}
\item September 5: Data encodings and steganography.
\item September 12: Disk and network forensics.
\item September 19: Classical cryptography.
\item September 26: Web security: XSS, CSRF.
\item October 3: Web security: SQL injection, command injection, misc.
\item October 10: x86 refresher, introduction to reverse-engineering.
\item October 17: Buffer overflows day one: zero protections.
\item October 24: Buffer overflows day two: NX \& ASLR.
\item October 31: Modern cryptography day one: hashing \& RSA.
\item November 7: Modern cryptography day two: side-channel attacks, CRIME.
\item November 14: Possibly no class
\item November 21: Exploitation day three: format strings.
\item December 5: TBA.
\end{enumerate}
\end{document}
